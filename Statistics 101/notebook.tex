
% Default to the notebook output style

    


% Inherit from the specified cell style.




    
\documentclass[11pt]{article}

    
    
    \usepackage[T1]{fontenc}
    % Nicer default font (+ math font) than Computer Modern for most use cases
    \usepackage{mathpazo}

    % Basic figure setup, for now with no caption control since it's done
    % automatically by Pandoc (which extracts ![](path) syntax from Markdown).
    \usepackage{graphicx}
    % We will generate all images so they have a width \maxwidth. This means
    % that they will get their normal width if they fit onto the page, but
    % are scaled down if they would overflow the margins.
    \makeatletter
    \def\maxwidth{\ifdim\Gin@nat@width>\linewidth\linewidth
    \else\Gin@nat@width\fi}
    \makeatother
    \let\Oldincludegraphics\includegraphics
    % Set max figure width to be 80% of text width, for now hardcoded.
    \renewcommand{\includegraphics}[1]{\Oldincludegraphics[width=.8\maxwidth]{#1}}
    % Ensure that by default, figures have no caption (until we provide a
    % proper Figure object with a Caption API and a way to capture that
    % in the conversion process - todo).
    \usepackage{caption}
    \DeclareCaptionLabelFormat{nolabel}{}
    \captionsetup{labelformat=nolabel}

    \usepackage{adjustbox} % Used to constrain images to a maximum size 
    \usepackage{xcolor} % Allow colors to be defined
    \usepackage{enumerate} % Needed for markdown enumerations to work
    \usepackage{geometry} % Used to adjust the document margins
    \usepackage{amsmath} % Equations
    \usepackage{amssymb} % Equations
    \usepackage{textcomp} % defines textquotesingle
    % Hack from http://tex.stackexchange.com/a/47451/13684:
    \AtBeginDocument{%
        \def\PYZsq{\textquotesingle}% Upright quotes in Pygmentized code
    }
    \usepackage{upquote} % Upright quotes for verbatim code
    \usepackage{eurosym} % defines \euro
    \usepackage[mathletters]{ucs} % Extended unicode (utf-8) support
    \usepackage[utf8x]{inputenc} % Allow utf-8 characters in the tex document
    \usepackage{fancyvrb} % verbatim replacement that allows latex
    \usepackage{grffile} % extends the file name processing of package graphics 
                         % to support a larger range 
    % The hyperref package gives us a pdf with properly built
    % internal navigation ('pdf bookmarks' for the table of contents,
    % internal cross-reference links, web links for URLs, etc.)
    \usepackage{hyperref}
    \usepackage{longtable} % longtable support required by pandoc >1.10
    \usepackage{booktabs}  % table support for pandoc > 1.12.2
    \usepackage[inline]{enumitem} % IRkernel/repr support (it uses the enumerate* environment)
    \usepackage[normalem]{ulem} % ulem is needed to support strikethroughs (\sout)
                                % normalem makes italics be italics, not underlines
    

    
    
    % Colors for the hyperref package
    \definecolor{urlcolor}{rgb}{0,.145,.698}
    \definecolor{linkcolor}{rgb}{.71,0.21,0.01}
    \definecolor{citecolor}{rgb}{.12,.54,.11}

    % ANSI colors
    \definecolor{ansi-black}{HTML}{3E424D}
    \definecolor{ansi-black-intense}{HTML}{282C36}
    \definecolor{ansi-red}{HTML}{E75C58}
    \definecolor{ansi-red-intense}{HTML}{B22B31}
    \definecolor{ansi-green}{HTML}{00A250}
    \definecolor{ansi-green-intense}{HTML}{007427}
    \definecolor{ansi-yellow}{HTML}{DDB62B}
    \definecolor{ansi-yellow-intense}{HTML}{B27D12}
    \definecolor{ansi-blue}{HTML}{208FFB}
    \definecolor{ansi-blue-intense}{HTML}{0065CA}
    \definecolor{ansi-magenta}{HTML}{D160C4}
    \definecolor{ansi-magenta-intense}{HTML}{A03196}
    \definecolor{ansi-cyan}{HTML}{60C6C8}
    \definecolor{ansi-cyan-intense}{HTML}{258F8F}
    \definecolor{ansi-white}{HTML}{C5C1B4}
    \definecolor{ansi-white-intense}{HTML}{A1A6B2}

    % commands and environments needed by pandoc snippets
    % extracted from the output of `pandoc -s`
    \providecommand{\tightlist}{%
      \setlength{\itemsep}{0pt}\setlength{\parskip}{0pt}}
    \DefineVerbatimEnvironment{Highlighting}{Verbatim}{commandchars=\\\{\}}
    % Add ',fontsize=\small' for more characters per line
    \newenvironment{Shaded}{}{}
    \newcommand{\KeywordTok}[1]{\textcolor[rgb]{0.00,0.44,0.13}{\textbf{{#1}}}}
    \newcommand{\DataTypeTok}[1]{\textcolor[rgb]{0.56,0.13,0.00}{{#1}}}
    \newcommand{\DecValTok}[1]{\textcolor[rgb]{0.25,0.63,0.44}{{#1}}}
    \newcommand{\BaseNTok}[1]{\textcolor[rgb]{0.25,0.63,0.44}{{#1}}}
    \newcommand{\FloatTok}[1]{\textcolor[rgb]{0.25,0.63,0.44}{{#1}}}
    \newcommand{\CharTok}[1]{\textcolor[rgb]{0.25,0.44,0.63}{{#1}}}
    \newcommand{\StringTok}[1]{\textcolor[rgb]{0.25,0.44,0.63}{{#1}}}
    \newcommand{\CommentTok}[1]{\textcolor[rgb]{0.38,0.63,0.69}{\textit{{#1}}}}
    \newcommand{\OtherTok}[1]{\textcolor[rgb]{0.00,0.44,0.13}{{#1}}}
    \newcommand{\AlertTok}[1]{\textcolor[rgb]{1.00,0.00,0.00}{\textbf{{#1}}}}
    \newcommand{\FunctionTok}[1]{\textcolor[rgb]{0.02,0.16,0.49}{{#1}}}
    \newcommand{\RegionMarkerTok}[1]{{#1}}
    \newcommand{\ErrorTok}[1]{\textcolor[rgb]{1.00,0.00,0.00}{\textbf{{#1}}}}
    \newcommand{\NormalTok}[1]{{#1}}
    
    % Additional commands for more recent versions of Pandoc
    \newcommand{\ConstantTok}[1]{\textcolor[rgb]{0.53,0.00,0.00}{{#1}}}
    \newcommand{\SpecialCharTok}[1]{\textcolor[rgb]{0.25,0.44,0.63}{{#1}}}
    \newcommand{\VerbatimStringTok}[1]{\textcolor[rgb]{0.25,0.44,0.63}{{#1}}}
    \newcommand{\SpecialStringTok}[1]{\textcolor[rgb]{0.73,0.40,0.53}{{#1}}}
    \newcommand{\ImportTok}[1]{{#1}}
    \newcommand{\DocumentationTok}[1]{\textcolor[rgb]{0.73,0.13,0.13}{\textit{{#1}}}}
    \newcommand{\AnnotationTok}[1]{\textcolor[rgb]{0.38,0.63,0.69}{\textbf{\textit{{#1}}}}}
    \newcommand{\CommentVarTok}[1]{\textcolor[rgb]{0.38,0.63,0.69}{\textbf{\textit{{#1}}}}}
    \newcommand{\VariableTok}[1]{\textcolor[rgb]{0.10,0.09,0.49}{{#1}}}
    \newcommand{\ControlFlowTok}[1]{\textcolor[rgb]{0.00,0.44,0.13}{\textbf{{#1}}}}
    \newcommand{\OperatorTok}[1]{\textcolor[rgb]{0.40,0.40,0.40}{{#1}}}
    \newcommand{\BuiltInTok}[1]{{#1}}
    \newcommand{\ExtensionTok}[1]{{#1}}
    \newcommand{\PreprocessorTok}[1]{\textcolor[rgb]{0.74,0.48,0.00}{{#1}}}
    \newcommand{\AttributeTok}[1]{\textcolor[rgb]{0.49,0.56,0.16}{{#1}}}
    \newcommand{\InformationTok}[1]{\textcolor[rgb]{0.38,0.63,0.69}{\textbf{\textit{{#1}}}}}
    \newcommand{\WarningTok}[1]{\textcolor[rgb]{0.38,0.63,0.69}{\textbf{\textit{{#1}}}}}
    
    
    % Define a nice break command that doesn't care if a line doesn't already
    % exist.
    \def\br{\hspace*{\fill} \\* }
    % Math Jax compatability definitions
    \def\gt{>}
    \def\lt{<}
    % Document parameters
    \title{2. Statistics 101 Descriptive Statistics}
    
    
    

    % Pygments definitions
    
\makeatletter
\def\PY@reset{\let\PY@it=\relax \let\PY@bf=\relax%
    \let\PY@ul=\relax \let\PY@tc=\relax%
    \let\PY@bc=\relax \let\PY@ff=\relax}
\def\PY@tok#1{\csname PY@tok@#1\endcsname}
\def\PY@toks#1+{\ifx\relax#1\empty\else%
    \PY@tok{#1}\expandafter\PY@toks\fi}
\def\PY@do#1{\PY@bc{\PY@tc{\PY@ul{%
    \PY@it{\PY@bf{\PY@ff{#1}}}}}}}
\def\PY#1#2{\PY@reset\PY@toks#1+\relax+\PY@do{#2}}

\expandafter\def\csname PY@tok@w\endcsname{\def\PY@tc##1{\textcolor[rgb]{0.73,0.73,0.73}{##1}}}
\expandafter\def\csname PY@tok@c\endcsname{\let\PY@it=\textit\def\PY@tc##1{\textcolor[rgb]{0.25,0.50,0.50}{##1}}}
\expandafter\def\csname PY@tok@cp\endcsname{\def\PY@tc##1{\textcolor[rgb]{0.74,0.48,0.00}{##1}}}
\expandafter\def\csname PY@tok@k\endcsname{\let\PY@bf=\textbf\def\PY@tc##1{\textcolor[rgb]{0.00,0.50,0.00}{##1}}}
\expandafter\def\csname PY@tok@kp\endcsname{\def\PY@tc##1{\textcolor[rgb]{0.00,0.50,0.00}{##1}}}
\expandafter\def\csname PY@tok@kt\endcsname{\def\PY@tc##1{\textcolor[rgb]{0.69,0.00,0.25}{##1}}}
\expandafter\def\csname PY@tok@o\endcsname{\def\PY@tc##1{\textcolor[rgb]{0.40,0.40,0.40}{##1}}}
\expandafter\def\csname PY@tok@ow\endcsname{\let\PY@bf=\textbf\def\PY@tc##1{\textcolor[rgb]{0.67,0.13,1.00}{##1}}}
\expandafter\def\csname PY@tok@nb\endcsname{\def\PY@tc##1{\textcolor[rgb]{0.00,0.50,0.00}{##1}}}
\expandafter\def\csname PY@tok@nf\endcsname{\def\PY@tc##1{\textcolor[rgb]{0.00,0.00,1.00}{##1}}}
\expandafter\def\csname PY@tok@nc\endcsname{\let\PY@bf=\textbf\def\PY@tc##1{\textcolor[rgb]{0.00,0.00,1.00}{##1}}}
\expandafter\def\csname PY@tok@nn\endcsname{\let\PY@bf=\textbf\def\PY@tc##1{\textcolor[rgb]{0.00,0.00,1.00}{##1}}}
\expandafter\def\csname PY@tok@ne\endcsname{\let\PY@bf=\textbf\def\PY@tc##1{\textcolor[rgb]{0.82,0.25,0.23}{##1}}}
\expandafter\def\csname PY@tok@nv\endcsname{\def\PY@tc##1{\textcolor[rgb]{0.10,0.09,0.49}{##1}}}
\expandafter\def\csname PY@tok@no\endcsname{\def\PY@tc##1{\textcolor[rgb]{0.53,0.00,0.00}{##1}}}
\expandafter\def\csname PY@tok@nl\endcsname{\def\PY@tc##1{\textcolor[rgb]{0.63,0.63,0.00}{##1}}}
\expandafter\def\csname PY@tok@ni\endcsname{\let\PY@bf=\textbf\def\PY@tc##1{\textcolor[rgb]{0.60,0.60,0.60}{##1}}}
\expandafter\def\csname PY@tok@na\endcsname{\def\PY@tc##1{\textcolor[rgb]{0.49,0.56,0.16}{##1}}}
\expandafter\def\csname PY@tok@nt\endcsname{\let\PY@bf=\textbf\def\PY@tc##1{\textcolor[rgb]{0.00,0.50,0.00}{##1}}}
\expandafter\def\csname PY@tok@nd\endcsname{\def\PY@tc##1{\textcolor[rgb]{0.67,0.13,1.00}{##1}}}
\expandafter\def\csname PY@tok@s\endcsname{\def\PY@tc##1{\textcolor[rgb]{0.73,0.13,0.13}{##1}}}
\expandafter\def\csname PY@tok@sd\endcsname{\let\PY@it=\textit\def\PY@tc##1{\textcolor[rgb]{0.73,0.13,0.13}{##1}}}
\expandafter\def\csname PY@tok@si\endcsname{\let\PY@bf=\textbf\def\PY@tc##1{\textcolor[rgb]{0.73,0.40,0.53}{##1}}}
\expandafter\def\csname PY@tok@se\endcsname{\let\PY@bf=\textbf\def\PY@tc##1{\textcolor[rgb]{0.73,0.40,0.13}{##1}}}
\expandafter\def\csname PY@tok@sr\endcsname{\def\PY@tc##1{\textcolor[rgb]{0.73,0.40,0.53}{##1}}}
\expandafter\def\csname PY@tok@ss\endcsname{\def\PY@tc##1{\textcolor[rgb]{0.10,0.09,0.49}{##1}}}
\expandafter\def\csname PY@tok@sx\endcsname{\def\PY@tc##1{\textcolor[rgb]{0.00,0.50,0.00}{##1}}}
\expandafter\def\csname PY@tok@m\endcsname{\def\PY@tc##1{\textcolor[rgb]{0.40,0.40,0.40}{##1}}}
\expandafter\def\csname PY@tok@gh\endcsname{\let\PY@bf=\textbf\def\PY@tc##1{\textcolor[rgb]{0.00,0.00,0.50}{##1}}}
\expandafter\def\csname PY@tok@gu\endcsname{\let\PY@bf=\textbf\def\PY@tc##1{\textcolor[rgb]{0.50,0.00,0.50}{##1}}}
\expandafter\def\csname PY@tok@gd\endcsname{\def\PY@tc##1{\textcolor[rgb]{0.63,0.00,0.00}{##1}}}
\expandafter\def\csname PY@tok@gi\endcsname{\def\PY@tc##1{\textcolor[rgb]{0.00,0.63,0.00}{##1}}}
\expandafter\def\csname PY@tok@gr\endcsname{\def\PY@tc##1{\textcolor[rgb]{1.00,0.00,0.00}{##1}}}
\expandafter\def\csname PY@tok@ge\endcsname{\let\PY@it=\textit}
\expandafter\def\csname PY@tok@gs\endcsname{\let\PY@bf=\textbf}
\expandafter\def\csname PY@tok@gp\endcsname{\let\PY@bf=\textbf\def\PY@tc##1{\textcolor[rgb]{0.00,0.00,0.50}{##1}}}
\expandafter\def\csname PY@tok@go\endcsname{\def\PY@tc##1{\textcolor[rgb]{0.53,0.53,0.53}{##1}}}
\expandafter\def\csname PY@tok@gt\endcsname{\def\PY@tc##1{\textcolor[rgb]{0.00,0.27,0.87}{##1}}}
\expandafter\def\csname PY@tok@err\endcsname{\def\PY@bc##1{\setlength{\fboxsep}{0pt}\fcolorbox[rgb]{1.00,0.00,0.00}{1,1,1}{\strut ##1}}}
\expandafter\def\csname PY@tok@kc\endcsname{\let\PY@bf=\textbf\def\PY@tc##1{\textcolor[rgb]{0.00,0.50,0.00}{##1}}}
\expandafter\def\csname PY@tok@kd\endcsname{\let\PY@bf=\textbf\def\PY@tc##1{\textcolor[rgb]{0.00,0.50,0.00}{##1}}}
\expandafter\def\csname PY@tok@kn\endcsname{\let\PY@bf=\textbf\def\PY@tc##1{\textcolor[rgb]{0.00,0.50,0.00}{##1}}}
\expandafter\def\csname PY@tok@kr\endcsname{\let\PY@bf=\textbf\def\PY@tc##1{\textcolor[rgb]{0.00,0.50,0.00}{##1}}}
\expandafter\def\csname PY@tok@bp\endcsname{\def\PY@tc##1{\textcolor[rgb]{0.00,0.50,0.00}{##1}}}
\expandafter\def\csname PY@tok@fm\endcsname{\def\PY@tc##1{\textcolor[rgb]{0.00,0.00,1.00}{##1}}}
\expandafter\def\csname PY@tok@vc\endcsname{\def\PY@tc##1{\textcolor[rgb]{0.10,0.09,0.49}{##1}}}
\expandafter\def\csname PY@tok@vg\endcsname{\def\PY@tc##1{\textcolor[rgb]{0.10,0.09,0.49}{##1}}}
\expandafter\def\csname PY@tok@vi\endcsname{\def\PY@tc##1{\textcolor[rgb]{0.10,0.09,0.49}{##1}}}
\expandafter\def\csname PY@tok@vm\endcsname{\def\PY@tc##1{\textcolor[rgb]{0.10,0.09,0.49}{##1}}}
\expandafter\def\csname PY@tok@sa\endcsname{\def\PY@tc##1{\textcolor[rgb]{0.73,0.13,0.13}{##1}}}
\expandafter\def\csname PY@tok@sb\endcsname{\def\PY@tc##1{\textcolor[rgb]{0.73,0.13,0.13}{##1}}}
\expandafter\def\csname PY@tok@sc\endcsname{\def\PY@tc##1{\textcolor[rgb]{0.73,0.13,0.13}{##1}}}
\expandafter\def\csname PY@tok@dl\endcsname{\def\PY@tc##1{\textcolor[rgb]{0.73,0.13,0.13}{##1}}}
\expandafter\def\csname PY@tok@s2\endcsname{\def\PY@tc##1{\textcolor[rgb]{0.73,0.13,0.13}{##1}}}
\expandafter\def\csname PY@tok@sh\endcsname{\def\PY@tc##1{\textcolor[rgb]{0.73,0.13,0.13}{##1}}}
\expandafter\def\csname PY@tok@s1\endcsname{\def\PY@tc##1{\textcolor[rgb]{0.73,0.13,0.13}{##1}}}
\expandafter\def\csname PY@tok@mb\endcsname{\def\PY@tc##1{\textcolor[rgb]{0.40,0.40,0.40}{##1}}}
\expandafter\def\csname PY@tok@mf\endcsname{\def\PY@tc##1{\textcolor[rgb]{0.40,0.40,0.40}{##1}}}
\expandafter\def\csname PY@tok@mh\endcsname{\def\PY@tc##1{\textcolor[rgb]{0.40,0.40,0.40}{##1}}}
\expandafter\def\csname PY@tok@mi\endcsname{\def\PY@tc##1{\textcolor[rgb]{0.40,0.40,0.40}{##1}}}
\expandafter\def\csname PY@tok@il\endcsname{\def\PY@tc##1{\textcolor[rgb]{0.40,0.40,0.40}{##1}}}
\expandafter\def\csname PY@tok@mo\endcsname{\def\PY@tc##1{\textcolor[rgb]{0.40,0.40,0.40}{##1}}}
\expandafter\def\csname PY@tok@ch\endcsname{\let\PY@it=\textit\def\PY@tc##1{\textcolor[rgb]{0.25,0.50,0.50}{##1}}}
\expandafter\def\csname PY@tok@cm\endcsname{\let\PY@it=\textit\def\PY@tc##1{\textcolor[rgb]{0.25,0.50,0.50}{##1}}}
\expandafter\def\csname PY@tok@cpf\endcsname{\let\PY@it=\textit\def\PY@tc##1{\textcolor[rgb]{0.25,0.50,0.50}{##1}}}
\expandafter\def\csname PY@tok@c1\endcsname{\let\PY@it=\textit\def\PY@tc##1{\textcolor[rgb]{0.25,0.50,0.50}{##1}}}
\expandafter\def\csname PY@tok@cs\endcsname{\let\PY@it=\textit\def\PY@tc##1{\textcolor[rgb]{0.25,0.50,0.50}{##1}}}

\def\PYZbs{\char`\\}
\def\PYZus{\char`\_}
\def\PYZob{\char`\{}
\def\PYZcb{\char`\}}
\def\PYZca{\char`\^}
\def\PYZam{\char`\&}
\def\PYZlt{\char`\<}
\def\PYZgt{\char`\>}
\def\PYZsh{\char`\#}
\def\PYZpc{\char`\%}
\def\PYZdl{\char`\$}
\def\PYZhy{\char`\-}
\def\PYZsq{\char`\'}
\def\PYZdq{\char`\"}
\def\PYZti{\char`\~}
% for compatibility with earlier versions
\def\PYZat{@}
\def\PYZlb{[}
\def\PYZrb{]}
\makeatother


    % Exact colors from NB
    \definecolor{incolor}{rgb}{0.0, 0.0, 0.5}
    \definecolor{outcolor}{rgb}{0.545, 0.0, 0.0}



    
    % Prevent overflowing lines due to hard-to-break entities
    \sloppy 
    % Setup hyperref package
    \hypersetup{
      breaklinks=true,  % so long urls are correctly broken across lines
      colorlinks=true,
      urlcolor=urlcolor,
      linkcolor=linkcolor,
      citecolor=citecolor,
      }
    % Slightly bigger margins than the latex defaults
    
    \geometry{verbose,tmargin=1in,bmargin=1in,lmargin=1in,rmargin=1in}
    
    

    \begin{document}
    
    
    \maketitle
    
    

    
    \begin{Verbatim}[commandchars=\\\{\}]
{\color{incolor}In [{\color{incolor}4}]:} \PY{k+kn}{import} \PY{n+nn}{pandas} \PY{k}{as} \PY{n+nn}{pd}
        \PY{k+kn}{import} \PY{n+nn}{seaborn} \PY{k}{as} \PY{n+nn}{sns}
        \PY{k+kn}{import} \PY{n+nn}{numpy} \PY{k}{as} \PY{n+nn}{np}
        \PY{k+kn}{import} \PY{n+nn}{matplotlib}\PY{n+nn}{.}\PY{n+nn}{pyplot} \PY{k}{as} \PY{n+nn}{plt}
        \PY{k+kn}{from} \PY{n+nn}{random} \PY{k}{import} \PY{n}{randint}
\end{Verbatim}


    \section{Describing a Categorical
Variable}\label{describing-a-categorical-variable}

    Categorical data use labels, names, or other desciptors to identify
exclusive categories or types of things - Region (North, South, East,
West) - Machine (M1, M2, M3) - Car Make (Ford, Toyota, Honda)

Quantitative Data are numerical values that represent frequency,
measurement, etc. - Sales for each region in millions (1.2, 3.5, 10.4,
6.9) - Production units (983, 1085, 899) - Fastest car mph (216, 156,
169)

    \subsection{Frequency Distribution}\label{frequency-distribution}

\subsubsection{Survey of 100 SmartPhone Users in the United
State}\label{survey-of-100-smartphone-users-in-the-united-state}

    \begin{Verbatim}[commandchars=\\\{\}]
{\color{incolor}In [{\color{incolor}3}]:} \PY{n}{phone\PYZus{}list} \PY{o}{=} \PY{p}{\PYZob{}}\PY{l+s+s1}{\PYZsq{}}\PY{l+s+s1}{SMARTPHONE BRAND}\PY{l+s+s1}{\PYZsq{}}\PY{p}{:} \PY{p}{[}\PY{l+s+s1}{\PYZsq{}}\PY{l+s+s1}{Apple}\PY{l+s+s1}{\PYZsq{}}\PY{p}{,} \PY{l+s+s1}{\PYZsq{}}\PY{l+s+s1}{HTC}\PY{l+s+s1}{\PYZsq{}}\PY{p}{,} \PY{l+s+s1}{\PYZsq{}}\PY{l+s+s1}{LG}\PY{l+s+s1}{\PYZsq{}}\PY{p}{,}\PY{l+s+s1}{\PYZsq{}}\PY{l+s+s1}{Motorola}\PY{l+s+s1}{\PYZsq{}}\PY{p}{,}\PY{l+s+s1}{\PYZsq{}}\PY{l+s+s1}{Samsung}\PY{l+s+s1}{\PYZsq{}}\PY{p}{,}\PY{l+s+s1}{\PYZsq{}}\PY{l+s+s1}{Other}\PY{l+s+s1}{\PYZsq{}}\PY{p}{]}\PY{p}{,} \PY{l+s+s1}{\PYZsq{}}\PY{l+s+s1}{FREQUENCY}\PY{l+s+s1}{\PYZsq{}}\PY{p}{:} \PY{p}{[}\PY{l+m+mi}{45}\PY{p}{,}\PY{l+m+mi}{2}\PY{p}{,}\PY{l+m+mi}{10}\PY{p}{,}\PY{l+m+mi}{4}\PY{p}{,}\PY{l+m+mi}{28}\PY{p}{,}\PY{l+m+mi}{11}\PY{p}{]}\PY{p}{\PYZcb{}}
        \PY{n}{df1} \PY{o}{=} \PY{n}{pd}\PY{o}{.}\PY{n}{DataFrame}\PY{p}{(}\PY{n}{data} \PY{o}{=} \PY{n}{phone\PYZus{}list}\PY{p}{)}
        \PY{n}{df1} \PY{o}{=} \PY{n}{df1}\PY{p}{[}\PY{p}{[}\PY{l+s+s1}{\PYZsq{}}\PY{l+s+s1}{SMARTPHONE BRAND}\PY{l+s+s1}{\PYZsq{}}\PY{p}{,} \PY{l+s+s1}{\PYZsq{}}\PY{l+s+s1}{FREQUENCY}\PY{l+s+s1}{\PYZsq{}}\PY{p}{]}\PY{p}{]}
        \PY{c+c1}{\PYZsh{}df1.set\PYZus{}index(\PYZsq{}SMARTPHONE BRAND\PYZsq{})}
        \PY{n}{df1}
\end{Verbatim}


\begin{Verbatim}[commandchars=\\\{\}]
{\color{outcolor}Out[{\color{outcolor}3}]:}   SMARTPHONE BRAND  FREQUENCY
        0            Apple         45
        1              HTC          2
        2               LG         10
        3         Motorola          4
        4          Samsung         28
        5            Other         11
\end{Verbatim}
            
    \begin{Verbatim}[commandchars=\\\{\}]
{\color{incolor}In [{\color{incolor}4}]:} \PY{n}{df2} \PY{o}{=} \PY{n}{pd}\PY{o}{.}\PY{n}{DataFrame}\PY{p}{(}\PY{p}{\PYZob{}}\PY{l+s+s1}{\PYZsq{}}\PY{l+s+s1}{SMARTPHONE BRAND}\PY{l+s+s1}{\PYZsq{}}\PY{p}{:} \PY{p}{[}\PY{l+s+s1}{\PYZsq{}}\PY{l+s+s1}{Total}\PY{l+s+s1}{\PYZsq{}}\PY{p}{]}\PY{p}{,} \PY{l+s+s1}{\PYZsq{}}\PY{l+s+s1}{FREQUENCY}\PY{l+s+s1}{\PYZsq{}}\PY{p}{:} \PY{p}{[}\PY{n}{df1}\PY{p}{[}\PY{l+s+s1}{\PYZsq{}}\PY{l+s+s1}{FREQUENCY}\PY{l+s+s1}{\PYZsq{}}\PY{p}{]}\PY{o}{.}\PY{n}{sum}\PY{p}{(}\PY{p}{)}\PY{p}{]}\PY{p}{\PYZcb{}}\PY{p}{)}
        \PY{n}{df1}\PY{o}{.}\PY{n}{append}\PY{p}{(}\PY{n}{df2}\PY{p}{,} \PY{n}{ignore\PYZus{}index}\PY{o}{=}\PY{k+kc}{True}\PY{p}{)}
\end{Verbatim}


\begin{Verbatim}[commandchars=\\\{\}]
{\color{outcolor}Out[{\color{outcolor}4}]:}    FREQUENCY SMARTPHONE BRAND
        0         45            Apple
        1          2              HTC
        2         10               LG
        3          4         Motorola
        4         28          Samsung
        5         11            Other
        6        100            Total
\end{Verbatim}
            
    \begin{Verbatim}[commandchars=\\\{\}]
{\color{incolor}In [{\color{incolor}5}]:} \PY{n}{ax} \PY{o}{=} \PY{n}{sns}\PY{o}{.}\PY{n}{barplot}\PY{p}{(}\PY{n}{x}\PY{o}{=}\PY{l+s+s2}{\PYZdq{}}\PY{l+s+s2}{SMARTPHONE BRAND}\PY{l+s+s2}{\PYZdq{}}\PY{p}{,} \PY{n}{y}\PY{o}{=}\PY{l+s+s2}{\PYZdq{}}\PY{l+s+s2}{FREQUENCY}\PY{l+s+s2}{\PYZdq{}}\PY{p}{,} \PY{n}{data}\PY{o}{=}\PY{n}{df1}\PY{p}{)}
\end{Verbatim}


    \begin{center}
    \adjustimage{max size={0.9\linewidth}{0.9\paperheight}}{output_6_0.png}
    \end{center}
    { \hspace*{\fill} \\}
    
    \subsection{Relative Frequency}\label{relative-frequency}

Relative Frequency of a class = \$ \frac{Frequency of the class}{n} \$

Relative Frequency "Samsung" = \$ \frac{28}{100} = 0.28 \$

    \begin{Verbatim}[commandchars=\\\{\}]
{\color{incolor}In [{\color{incolor}6}]:} \PY{c+c1}{\PYZsh{}\PYZsh{} Frequency Distribution}
        \PY{n}{df1}\PY{p}{[}\PY{l+s+s1}{\PYZsq{}}\PY{l+s+s1}{RELATIVE FREQUENCY}\PY{l+s+s1}{\PYZsq{}}\PY{p}{]} \PY{o}{=} \PY{n}{df1}\PY{p}{[}\PY{l+s+s1}{\PYZsq{}}\PY{l+s+s1}{FREQUENCY}\PY{l+s+s1}{\PYZsq{}}\PY{p}{]}\PY{o}{.}\PY{n}{apply}\PY{p}{(}\PY{k}{lambda} \PY{n}{x}\PY{p}{:} \PY{n}{x}\PY{o}{/}\PY{l+m+mi}{100}\PY{p}{)}
        \PY{n}{df1} \PY{o}{=} \PY{n}{df1}\PY{p}{[}\PY{p}{[}\PY{l+s+s1}{\PYZsq{}}\PY{l+s+s1}{SMARTPHONE BRAND}\PY{l+s+s1}{\PYZsq{}}\PY{p}{,} \PY{l+s+s1}{\PYZsq{}}\PY{l+s+s1}{FREQUENCY}\PY{l+s+s1}{\PYZsq{}}\PY{p}{,} \PY{l+s+s1}{\PYZsq{}}\PY{l+s+s1}{RELATIVE FREQUENCY}\PY{l+s+s1}{\PYZsq{}}\PY{p}{]}\PY{p}{]}
        \PY{n}{df1}
\end{Verbatim}


\begin{Verbatim}[commandchars=\\\{\}]
{\color{outcolor}Out[{\color{outcolor}6}]:}   SMARTPHONE BRAND  FREQUENCY  RELATIVE FREQUENCY
        0            Apple         45                0.45
        1              HTC          2                0.02
        2               LG         10                0.10
        3         Motorola          4                0.04
        4          Samsung         28                0.28
        5            Other         11                0.11
\end{Verbatim}
            
    \begin{Verbatim}[commandchars=\\\{\}]
{\color{incolor}In [{\color{incolor}7}]:} \PY{n}{ax} \PY{o}{=} \PY{n}{sns}\PY{o}{.}\PY{n}{barplot}\PY{p}{(}\PY{n}{x}\PY{o}{=}\PY{l+s+s2}{\PYZdq{}}\PY{l+s+s2}{SMARTPHONE BRAND}\PY{l+s+s2}{\PYZdq{}}\PY{p}{,} \PY{n}{y}\PY{o}{=}\PY{l+s+s2}{\PYZdq{}}\PY{l+s+s2}{RELATIVE FREQUENCY}\PY{l+s+s2}{\PYZdq{}}\PY{p}{,} \PY{n}{data}\PY{o}{=}\PY{n}{df1}\PY{p}{)}
\end{Verbatim}


    \begin{center}
    \adjustimage{max size={0.9\linewidth}{0.9\paperheight}}{output_9_0.png}
    \end{center}
    { \hspace*{\fill} \\}
    
    \section{Descriptive Statistics,
Histograms}\label{descriptive-statistics-histograms}

\subsubsection{Survey of 100 SmartPhone Users in the United States -
AGE}\label{survey-of-100-smartphone-users-in-the-united-states---age}

    Histograms divide the entire range of values into a series of intervals.

Ex Age "bin" 18+ of 100 people seperates into three different bins -
18-39 has 39 people - 40-59 has 45 people - 60+ has 16 people

    \begin{itemize}
\item
  Too few bins can create a histogram that doesn't show the shape
  distribution of the underlying data
\item
  Too many bins create a histogram where there are too few observations
  in each bin and the overall general shape distribution is too broken
  up
\item
  Arguably the most useful preliminary visualization for quantitative
  data
\item
  Shows the shape of the distribution of values
\item
  Horizontal (x-axis) is the variable of interest (AGE)
\item
  Vertical (y-axis) can summarize the frequency, relative frequency, or
  percent frequency
\item
  Vertical rectangle for each class or bin
\item
  Height is determined by frequency, relative frequency, or percent
  frequency
\item
  No space or gaps between bars of a histogram
\end{itemize}

    \begin{Verbatim}[commandchars=\\\{\}]
{\color{incolor}In [{\color{incolor}8}]:} \PY{n}{ages\PYZus{}of\PYZus{}hundred} \PY{o}{=} \PY{n}{pd}\PY{o}{.}\PY{n}{Series}\PY{p}{(}\PY{p}{\PYZob{}}\PY{l+s+s1}{\PYZsq{}}\PY{l+s+s1}{Ages}\PY{l+s+s1}{\PYZsq{}}\PY{p}{:} \PY{n}{np}\PY{o}{.}\PY{n}{random}\PY{o}{.}\PY{n}{randint}\PY{p}{(}\PY{l+m+mi}{18}\PY{p}{,} \PY{l+m+mi}{99}\PY{p}{,} \PY{l+m+mi}{100}\PY{p}{)}\PY{p}{\PYZcb{}}\PY{p}{)}
        \PY{n+nb}{print}\PY{p}{(}\PY{n}{ages\PYZus{}of\PYZus{}hundred}\PY{o}{.}\PY{n}{Ages}\PY{p}{)}
        \PY{n}{ages\PYZus{}of\PYZus{}hundred}\PY{p}{[}\PY{l+s+s1}{\PYZsq{}}\PY{l+s+s1}{Ages}\PY{l+s+s1}{\PYZsq{}}\PY{p}{]}\PY{o}{.}\PY{n}{size}
        \PY{n}{first\PYZus{}bin} \PY{o}{=} \PY{n}{ages\PYZus{}of\PYZus{}hundred}\PY{o}{.}\PY{n}{Ages}\PY{p}{[}\PY{p}{(}\PY{n}{ages\PYZus{}of\PYZus{}hundred}\PY{o}{.}\PY{n}{Ages} \PY{o}{\PYZgt{}} \PY{l+m+mi}{17}\PY{p}{)} \PY{o}{\PYZam{}} \PY{p}{(}\PY{n}{ages\PYZus{}of\PYZus{}hundred}\PY{o}{.}\PY{n}{Ages} \PY{o}{\PYZlt{}} \PY{l+m+mi}{40}\PY{p}{)}\PY{p}{]}\PY{o}{.}\PY{n}{size}
        \PY{n}{second\PYZus{}bin} \PY{o}{=} \PY{n}{ages\PYZus{}of\PYZus{}hundred}\PY{o}{.}\PY{n}{Ages}\PY{p}{[}\PY{p}{(}\PY{n}{ages\PYZus{}of\PYZus{}hundred}\PY{o}{.}\PY{n}{Ages} \PY{o}{\PYZgt{}} \PY{l+m+mi}{39}\PY{p}{)} \PY{o}{\PYZam{}} \PY{p}{(}\PY{n}{ages\PYZus{}of\PYZus{}hundred}\PY{o}{.}\PY{n}{Ages} \PY{o}{\PYZlt{}} \PY{l+m+mi}{60}\PY{p}{)}\PY{p}{]}\PY{o}{.}\PY{n}{size}
        \PY{n}{third\PYZus{}bin} \PY{o}{=} \PY{n}{ages\PYZus{}of\PYZus{}hundred}\PY{o}{.}\PY{n}{Ages}\PY{p}{[}\PY{n}{ages\PYZus{}of\PYZus{}hundred}\PY{o}{.}\PY{n}{Ages} \PY{o}{\PYZgt{}} \PY{l+m+mi}{59}\PY{p}{]}\PY{o}{.}\PY{n}{size}
\end{Verbatim}


    \begin{Verbatim}[commandchars=\\\{\}]
[62 91 91 40 97 38 32 57 86 40 91 26 94 93 42 92 80 73 71 66 87 24 29 76
 20 26 55 52 93 88 90 63 32 47 44 88 97 69 46 22 61 82 37 39 92 70 79 57
 80 28 32 98 23 31 90 18 76 34 86 95 96 78 20 25 82 71 47 93 78 67 71 61
 83 90 30 70 75 48 27 95 54 35 70 24 70 49 46 50 29 61 23 45 46 69 70 90
 73 63 21 42]

    \end{Verbatim}

    \begin{Verbatim}[commandchars=\\\{\}]
{\color{incolor}In [{\color{incolor}55}]:} \PY{n}{cat} \PY{o}{=} \PY{p}{[}\PY{l+s+s1}{\PYZsq{}}\PY{l+s+s1}{18\PYZhy{}39}\PY{l+s+s1}{\PYZsq{}}\PY{p}{,}\PY{l+s+s1}{\PYZsq{}}\PY{l+s+s1}{40\PYZhy{}59}\PY{l+s+s1}{\PYZsq{}}\PY{p}{,}\PY{l+s+s1}{\PYZsq{}}\PY{l+s+s1}{60+}\PY{l+s+s1}{\PYZsq{}}\PY{p}{]}
         \PY{n}{df\PYZus{}ages} \PY{o}{=} \PY{n}{pd}\PY{o}{.}\PY{n}{DataFrame}\PY{p}{(}\PY{p}{\PYZob{}}\PY{l+s+s1}{\PYZsq{}}\PY{l+s+s1}{Ages}\PY{l+s+s1}{\PYZsq{}}\PY{p}{:} \PY{p}{[}\PY{n}{cat}\PY{p}{[}\PY{l+m+mi}{0}\PY{p}{]}\PY{p}{,} \PY{n}{cat}\PY{p}{[}\PY{l+m+mi}{1}\PY{p}{]}\PY{p}{,} \PY{n}{cat}\PY{p}{[}\PY{l+m+mi}{2}\PY{p}{]}\PY{p}{]}\PY{p}{,} \PY{l+s+s1}{\PYZsq{}}\PY{l+s+s1}{Frequency}\PY{l+s+s1}{\PYZsq{}}\PY{p}{:} \PY{p}{[}\PY{n}{first\PYZus{}bin}\PY{p}{,} \PY{n}{second\PYZus{}bin}\PY{p}{,} \PY{n}{third\PYZus{}bin}\PY{p}{]}\PY{p}{\PYZcb{}}\PY{p}{)}
         \PY{n}{df\PYZus{}ages}
\end{Verbatim}


\begin{Verbatim}[commandchars=\\\{\}]
{\color{outcolor}Out[{\color{outcolor}55}]:}     Ages  Frequency
         0  18-39         26
         1  40-59         19
         2    60+         55
\end{Verbatim}
            
    \begin{Verbatim}[commandchars=\\\{\}]
{\color{incolor}In [{\color{incolor}56}]:} \PY{n}{ax} \PY{o}{=} \PY{n}{sns}\PY{o}{.}\PY{n}{barplot}\PY{p}{(}\PY{n}{x}\PY{o}{=}\PY{l+s+s2}{\PYZdq{}}\PY{l+s+s2}{Ages}\PY{l+s+s2}{\PYZdq{}}\PY{p}{,} \PY{n}{y}\PY{o}{=}\PY{l+s+s2}{\PYZdq{}}\PY{l+s+s2}{Frequency}\PY{l+s+s2}{\PYZdq{}}\PY{p}{,} \PY{n}{data}\PY{o}{=}\PY{n}{df\PYZus{}ages}\PY{p}{)}
\end{Verbatim}


    \begin{center}
    \adjustimage{max size={0.9\linewidth}{0.9\paperheight}}{output_15_0.png}
    \end{center}
    { \hspace*{\fill} \\}
    
    \begin{Verbatim}[commandchars=\\\{\}]
{\color{incolor}In [{\color{incolor}57}]:} \PY{k}{def} \PY{n+nf}{ages\PYZus{}cat}\PY{p}{(}\PY{n}{col}\PY{p}{)}\PY{p}{:}
             \PY{k}{if} \PY{p}{(}\PY{n}{col}\PY{o}{\PYZgt{}}\PY{l+m+mi}{17}\PY{p}{)} \PY{o}{\PYZam{}} \PY{p}{(}\PY{n}{col}\PY{o}{\PYZlt{}}\PY{l+m+mi}{40}\PY{p}{)}\PY{p}{:}
                 \PY{k}{return} \PY{l+s+s1}{\PYZsq{}}\PY{l+s+s1}{18\PYZhy{}39}\PY{l+s+s1}{\PYZsq{}}
             \PY{k}{elif} \PY{p}{(}\PY{n}{col}\PY{o}{\PYZgt{}}\PY{l+m+mi}{39}\PY{p}{)} \PY{o}{\PYZam{}} \PY{p}{(}\PY{n}{col}\PY{o}{\PYZlt{}}\PY{l+m+mi}{60}\PY{p}{)}\PY{p}{:}
                 \PY{k}{return} \PY{l+s+s1}{\PYZsq{}}\PY{l+s+s1}{40\PYZhy{}59}\PY{l+s+s1}{\PYZsq{}}
             \PY{k}{elif} \PY{n}{col}\PY{o}{\PYZgt{}}\PY{l+m+mi}{59}\PY{p}{:}
                 \PY{k}{return} \PY{l+s+s1}{\PYZsq{}}\PY{l+s+s1}{60+}\PY{l+s+s1}{\PYZsq{}}
             
         \PY{n}{ages\PYZus{}of\PYZus{}hundred} \PY{o}{=} \PY{n}{pd}\PY{o}{.}\PY{n}{DataFrame}\PY{p}{(}\PY{p}{\PYZob{}}\PY{l+s+s1}{\PYZsq{}}\PY{l+s+s1}{Frequency}\PY{l+s+s1}{\PYZsq{}}\PY{p}{:} \PY{n}{np}\PY{o}{.}\PY{n}{random}\PY{o}{.}\PY{n}{randint}\PY{p}{(}\PY{l+m+mi}{18}\PY{p}{,} \PY{l+m+mi}{99}\PY{p}{,} \PY{l+m+mi}{100}\PY{p}{)}\PY{p}{\PYZcb{}}\PY{p}{)}
         \PY{n}{ages\PYZus{}of\PYZus{}hundred}\PY{p}{[}\PY{l+s+s1}{\PYZsq{}}\PY{l+s+s1}{Age\PYZus{}Category}\PY{l+s+s1}{\PYZsq{}}\PY{p}{]} \PY{o}{=} \PY{n}{ages\PYZus{}of\PYZus{}hundred}\PY{p}{[}\PY{l+s+s1}{\PYZsq{}}\PY{l+s+s1}{Frequency}\PY{l+s+s1}{\PYZsq{}}\PY{p}{]}\PY{o}{.}\PY{n}{apply}\PY{p}{(}\PY{n}{ages\PYZus{}cat}\PY{p}{)}
         \PY{n}{ages\PYZus{}df} \PY{o}{=} \PY{n}{ages\PYZus{}of\PYZus{}hundred}\PY{p}{[}\PY{l+s+s1}{\PYZsq{}}\PY{l+s+s1}{Frequency}\PY{l+s+s1}{\PYZsq{}}\PY{p}{]}\PY{o}{.}\PY{n}{groupby}\PY{p}{(}\PY{n}{ages\PYZus{}of\PYZus{}hundred}\PY{p}{[}\PY{l+s+s1}{\PYZsq{}}\PY{l+s+s1}{Age\PYZus{}Category}\PY{l+s+s1}{\PYZsq{}}\PY{p}{]}\PY{p}{)}\PY{o}{.}\PY{n}{count}\PY{p}{(}\PY{p}{)}
         \PY{n}{ages\PYZus{}df} \PY{o}{=} \PY{n}{ages\PYZus{}df}\PY{o}{.}\PY{n}{to\PYZus{}frame}\PY{p}{(}\PY{p}{)}\PY{o}{.}\PY{n}{reset\PYZus{}index}\PY{p}{(}\PY{p}{)}
         \PY{n}{ages\PYZus{}df}
\end{Verbatim}


\begin{Verbatim}[commandchars=\\\{\}]
{\color{outcolor}Out[{\color{outcolor}57}]:}   Age\_Category  Frequency
         0        18-39         29
         1        40-59         21
         2          60+         50
\end{Verbatim}
            
    \begin{Verbatim}[commandchars=\\\{\}]
{\color{incolor}In [{\color{incolor}58}]:} \PY{n}{ax} \PY{o}{=} \PY{n}{sns}\PY{o}{.}\PY{n}{barplot}\PY{p}{(}\PY{n}{x}\PY{o}{=}\PY{l+s+s2}{\PYZdq{}}\PY{l+s+s2}{Age\PYZus{}Category}\PY{l+s+s2}{\PYZdq{}}\PY{p}{,} \PY{n}{y}\PY{o}{=}\PY{l+s+s2}{\PYZdq{}}\PY{l+s+s2}{Frequency}\PY{l+s+s2}{\PYZdq{}}\PY{p}{,} \PY{n}{data}\PY{o}{=}\PY{n}{ages\PYZus{}df}\PY{p}{)}
\end{Verbatim}


    \begin{center}
    \adjustimage{max size={0.9\linewidth}{0.9\paperheight}}{output_17_0.png}
    \end{center}
    { \hspace*{\fill} \\}
    
    \section{Stem and Leaf Diagrams}\label{stem-and-leaf-diagrams}

    \begin{itemize}
\item
  A stem and leaf display shows
\item
  \begin{itemize}
  \tightlist
  \item
    The rank order of the data
  \end{itemize}
\item
  \begin{itemize}
  \tightlist
  \item
    The shape of the distribution of data
  \end{itemize}
\item
  \begin{itemize}
  \tightlist
  \item
    Modal qualities of the data (frequent observations)
  \end{itemize}
\item
  It's called a stem and leaf because that is what it looks like; a stem
  with leaves off the side
\item
  It takes the form of STEM \textbar{} LEAF
\item
  The leaf is always the last single digit
\item
  The stem is formed from all digits left of the last digit
\end{itemize}

    \begin{Verbatim}[commandchars=\\\{\}]
{\color{incolor}In [{\color{incolor}6}]:} \PY{n}{stem\PYZus{}array} \PY{o}{=} \PY{n}{np}\PY{o}{.}\PY{n}{array}\PY{p}{(}\PY{p}{[}\PY{l+m+mi}{50}\PY{p}{,}\PY{l+m+mi}{61}\PY{p}{,}\PY{l+m+mi}{66}\PY{p}{,}\PY{l+m+mi}{73}\PY{p}{,}\PY{l+m+mi}{82}\PY{p}{,}\PY{l+m+mi}{82}\PY{p}{,}\PY{l+m+mi}{90}\PY{p}{,}\PY{l+m+mi}{103}\PY{p}{,}\PY{l+m+mi}{108}\PY{p}{,}\PY{l+m+mi}{115}\PY{p}{]}\PY{p}{)}
\end{Verbatim}


    \begin{matrix}
  Stem & | & Leaves\\
  5 & | & 0\\
  6 & | & 1 & 6\\
  7 & | & 3\\
  8 & | & 2 & 2\\
  9 & | & 0\\
  10 & | & 3 & 8\\
  11 & | & 5\\
\end{matrix}

    \section{Descriptive Statistics Cross
Tabulation}\label{descriptive-statistics-cross-tabulation}

    \begin{itemize}
\tightlist
\item
  Crosstabulation is a table summary for two variables
\item
  Shows relationship between two variables
\item
  The size of the cross tabulation is the number of categories of one
  variable multiplied by the number of categories of the second variable
\item
  If the first category has 5 categories and the second variable has 6
  categories then the crosstabulation if 5 * 6 = 30 cells
\item
  Each cell is an intersection of the two categories at that point
\item
  Can be done using pivot tables
\end{itemize}

    Example:

4 Regions - North, South, East, West 5 product Segments - Smartphones,
Televisions, Gaming, Appliances, Computers 100 observations of sales for
each region X segment combination

    \begin{Verbatim}[commandchars=\\\{\}]
{\color{incolor}In [{\color{incolor}210}]:} \PY{n}{sales} \PY{o}{=} \PY{n}{pd}\PY{o}{.}\PY{n}{read\PYZus{}csv}\PY{p}{(}\PY{l+s+s1}{\PYZsq{}}\PY{l+s+s1}{sales.csv}\PY{l+s+s1}{\PYZsq{}}\PY{p}{,} \PY{n}{sep} \PY{o}{=} \PY{l+s+s1}{\PYZsq{}}\PY{l+s+s1}{,}\PY{l+s+s1}{\PYZsq{}}\PY{p}{)}
          \PY{n+nb}{print}\PY{p}{(}\PY{n}{sales}\PY{p}{,}\PY{n}{sales}\PY{o}{.}\PY{n}{dtypes}\PY{p}{)}
          \PY{n+nb}{type}\PY{p}{(}\PY{n}{sales}\PY{p}{)}
\end{Verbatim}


    \begin{Verbatim}[commandchars=\\\{\}]
   Region      Product  Price
0    East   Appliances   1704
1   North   Appliances   1656
2   South   Appliances   1342
3    West   Appliances   1243
4    East    Computers   1586
5   North    Computers   1583
6   South    Computers   1152
7    West    Computers   1382
8    East       Gaming   1681
9   North       Gaming   1859
10  South       Gaming   1367
11   West       Gaming   1605
12   East  Smartphones   1341
13  North  Smartphones   1336
14  South  Smartphones   1252
15   West  Smartphones   1785
16   East  Televisions   1415
17  North  Televisions   1128
18  South  Televisions   1709
19   West  Televisions   1572 Region     object
Product    object
Price       int64
dtype: object

    \end{Verbatim}

\begin{Verbatim}[commandchars=\\\{\}]
{\color{outcolor}Out[{\color{outcolor}210}]:} pandas.core.frame.DataFrame
\end{Verbatim}
            
    \begin{Verbatim}[commandchars=\\\{\}]
{\color{incolor}In [{\color{incolor}179}]:} \PY{n}{sales\PYZus{}by\PYZus{}region} \PY{o}{=} \PY{n}{pd}\PY{o}{.}\PY{n}{pivot\PYZus{}table}\PY{p}{(}\PY{n}{sales}\PY{p}{,} \PY{n}{values}\PY{o}{=}\PY{l+s+s1}{\PYZsq{}}\PY{l+s+s1}{Price}\PY{l+s+s1}{\PYZsq{}}\PY{p}{,} \PY{n}{index}\PY{o}{=}\PY{p}{[}\PY{l+s+s1}{\PYZsq{}}\PY{l+s+s1}{Region}\PY{l+s+s1}{\PYZsq{}}\PY{p}{]}\PY{p}{,} \PY{n}{columns}\PY{o}{=}\PY{p}{[}\PY{l+s+s1}{\PYZsq{}}\PY{l+s+s1}{Product}\PY{l+s+s1}{\PYZsq{}}\PY{p}{]}\PY{p}{,} \PY{n}{aggfunc}\PY{o}{=}\PY{n}{np}\PY{o}{.}\PY{n}{sum}\PY{p}{)}
          \PY{n}{sales\PYZus{}by\PYZus{}region}
\end{Verbatim}


\begin{Verbatim}[commandchars=\\\{\}]
{\color{outcolor}Out[{\color{outcolor}179}]:} Product  Appliances  Computers  Gaming  Smartphones  Televisions
          Region                                                          
          East           1704       1586    1681         1341         1415
          North          1656       1583    1859         1336         1128
          South          1342       1152    1367         1252         1709
          West           1243       1382    1605         1785         1572
\end{Verbatim}
            
    \begin{Verbatim}[commandchars=\\\{\}]
{\color{incolor}In [{\color{incolor}180}]:} \PY{n}{sales\PYZus{}by\PYZus{}region}\PY{p}{[}\PY{l+s+s1}{\PYZsq{}}\PY{l+s+s1}{Total}\PY{l+s+s1}{\PYZsq{}}\PY{p}{]} \PY{o}{=} \PY{n}{sales\PYZus{}by\PYZus{}region}\PY{o}{.}\PY{n}{apply}\PY{p}{(}\PY{n}{np}\PY{o}{.}\PY{n}{sum}\PY{p}{,} \PY{n}{axis}\PY{o}{=}\PY{l+m+mi}{1}\PY{p}{)}
          \PY{n}{sales\PYZus{}by\PYZus{}region}\PY{o}{.}\PY{n}{reset\PYZus{}index}\PY{p}{(}\PY{p}{)}
\end{Verbatim}


\begin{Verbatim}[commandchars=\\\{\}]
{\color{outcolor}Out[{\color{outcolor}180}]:} Product Region  Appliances  Computers  Gaming  Smartphones  Televisions  Total
          0         East        1704       1586    1681         1341         1415   7727
          1        North        1656       1583    1859         1336         1128   7562
          2        South        1342       1152    1367         1252         1709   6822
          3         West        1243       1382    1605         1785         1572   7587
\end{Verbatim}
            
    \begin{Verbatim}[commandchars=\\\{\}]
{\color{incolor}In [{\color{incolor}186}]:} \PY{n}{total} \PY{o}{=} \PY{n}{sales\PYZus{}by\PYZus{}region}\PY{o}{.}\PY{n}{apply}\PY{p}{(}\PY{n}{np}\PY{o}{.}\PY{n}{sum}\PY{p}{,} \PY{n}{axis}\PY{o}{=}\PY{l+m+mi}{0}\PY{p}{)}
          \PY{n}{total}\PY{o}{.}\PY{n}{values}
          \PY{n}{total}\PY{o}{.}\PY{n}{keys}\PY{p}{(}\PY{p}{)}
          \PY{n}{values} \PY{o}{=} \PY{n}{total}\PY{o}{.}\PY{n}{values} \PY{c+c1}{\PYZsh{}np.insert(total.values, 0, 0)}
          \PY{n}{keys} \PY{o}{=} \PY{n}{total}\PY{o}{.}\PY{n}{keys}\PY{p}{(}\PY{p}{)}
          \PY{n}{total\PYZus{}df} \PY{o}{=} \PY{n}{pd}\PY{o}{.}\PY{n}{DataFrame}\PY{p}{(}\PY{n}{values}\PY{p}{,} \PY{n}{index}\PY{o}{=}\PY{n}{keys}\PY{p}{)}\PY{o}{.}\PY{n}{T}
          \PY{n}{sales\PYZus{}by\PYZus{}region1} \PY{o}{=} \PY{n}{sales\PYZus{}by\PYZus{}region}\PY{o}{.}\PY{n}{append}\PY{p}{(}\PY{n}{total\PYZus{}df}\PY{p}{)}
          \PY{n}{sales\PYZus{}by\PYZus{}region1}\PY{o}{.}\PY{n}{rename}\PY{p}{(}\PY{n}{index}\PY{o}{=}\PY{p}{\PYZob{}}\PY{l+m+mi}{0}\PY{p}{:} \PY{l+s+s1}{\PYZsq{}}\PY{l+s+s1}{Total}\PY{l+s+s1}{\PYZsq{}}\PY{p}{\PYZcb{}}\PY{p}{)}
\end{Verbatim}


\begin{Verbatim}[commandchars=\\\{\}]
{\color{outcolor}Out[{\color{outcolor}186}]:} Product  Appliances  Computers  Gaming  Smartphones  Televisions  Total
          East           1704       1586    1681         1341         1415   7727
          North          1656       1583    1859         1336         1128   7562
          South          1342       1152    1367         1252         1709   6822
          West           1243       1382    1605         1785         1572   7587
          Total          5945       5703    6512         5714         5824  29698
\end{Verbatim}
            
    \begin{Verbatim}[commandchars=\\\{\}]
{\color{incolor}In [{\color{incolor}192}]:} \PY{n}{sales\PYZus{}by\PYZus{}region1} \PY{o}{=} \PY{n}{pd}\PY{o}{.}\PY{n}{pivot\PYZus{}table}\PY{p}{(}\PY{n}{sales}\PY{p}{,} \PY{n}{values}\PY{o}{=}\PY{l+s+s1}{\PYZsq{}}\PY{l+s+s1}{Price}\PY{l+s+s1}{\PYZsq{}}\PY{p}{,} \PY{n}{index}\PY{o}{=}\PY{p}{[}\PY{l+s+s1}{\PYZsq{}}\PY{l+s+s1}{Region}\PY{l+s+s1}{\PYZsq{}}\PY{p}{]}\PY{p}{,} \PY{n}{columns}\PY{o}{=}\PY{p}{[}\PY{l+s+s1}{\PYZsq{}}\PY{l+s+s1}{Product}\PY{l+s+s1}{\PYZsq{}}\PY{p}{]}\PY{p}{,} \PY{n}{aggfunc}\PY{o}{=}\PY{n}{np}\PY{o}{.}\PY{n}{sum}\PY{p}{)}
          \PY{n}{sales\PYZus{}by\PYZus{}region1}\PY{p}{[}\PY{l+s+s1}{\PYZsq{}}\PY{l+s+s1}{Total}\PY{l+s+s1}{\PYZsq{}}\PY{p}{]} \PY{o}{=} \PY{n}{sales\PYZus{}by\PYZus{}region}\PY{o}{.}\PY{n}{apply}\PY{p}{(}\PY{n}{np}\PY{o}{.}\PY{n}{sum}\PY{p}{,} \PY{n}{axis}\PY{o}{=}\PY{l+m+mi}{1}\PY{p}{)}
          \PY{n}{sales\PYZus{}by\PYZus{}region1}\PY{o}{.}\PY{n}{loc}\PY{p}{[}\PY{l+s+s1}{\PYZsq{}}\PY{l+s+s1}{Total}\PY{l+s+s1}{\PYZsq{}}\PY{p}{]} \PY{o}{=} \PY{n}{sales\PYZus{}by\PYZus{}region}\PY{o}{.}\PY{n}{apply}\PY{p}{(}\PY{n}{np}\PY{o}{.}\PY{n}{sum}\PY{p}{,} \PY{n}{axis}\PY{o}{=}\PY{l+m+mi}{0}\PY{p}{)}
          \PY{n}{sales\PYZus{}by\PYZus{}region1}
\end{Verbatim}


\begin{Verbatim}[commandchars=\\\{\}]
{\color{outcolor}Out[{\color{outcolor}192}]:} Product  Appliances  Computers  Gaming  Smartphones  Televisions  Total
          Region                                                                 
          East           1704       1586    1681         1341         1415  15454
          North          1656       1583    1859         1336         1128  15124
          South          1342       1152    1367         1252         1709  13644
          West           1243       1382    1605         1785         1572  15174
          Total          5945       5703    6512         5714         5824  29698
\end{Verbatim}
            

    % Add a bibliography block to the postdoc
    
    
    
    \end{document}
